\documentclass[a4paper,AutoFakeBold={2.7}]{ctexart} %纸张大小A4,文本类型为ctexart
%--------------------导言区:加载宏包,改变全局设置等--------------------%
\usepackage[colorlinks=true,urlcolor=black,linkcolor=black,citecolor=black]{hyperref}
%超链接宏包,消除外部网址链接,文内交叉引用链接,参考文献引用链接的方框,且将字色改为黑
\usepackage{amsmath}%数学宏包1
\usepackage{amssymb}%数学宏包2
\usepackage{arydshln}%分块矩阵宏包
\usepackage{fontspec}%字体设置宏包
\usepackage{booktabs}%三线表宏包
\usepackage{array}%表格宏包
\usepackage{caption}%题注设置宏包
\usepackage{graphicx}%图片宏包
\usepackage{float}%浮动体宏包,强制让浮动体在Here位置
\usepackage{subfig}%子图宏包
\usepackage[left=2.5cm,right=2.5cm,top=2.5cm,bottom=2.5cm]{geometry}%页边距设置
\usepackage{wrapfig}%图片文字环绕宏包
\usepackage{listings}%代码高亮宏包
\usepackage{xcolor}%颜色处理宏包
\DeclareCaptionFont{heiti}{\heiti}%为caption宏包定义heiti命令
\captionsetup{labelsep=space,font={small,bf,heiti},skip=5pt}%题注设置
\setmainfont{Times New Roman}%英文设置为Times New Roman
\renewcommand{\baselinestretch}{1.25}%改变默认行距为1.25倍行距
\pagestyle{plain}%没有页眉,页脚包含页码
\ctexset{section={format={\heiti \zihao{-3} \bfseries}},
	subsection={format={\heiti \zihao{4} \bfseries},beforeskip=0pt,afterskip=0pt},
	subsubsection={format={\heiti \zihao{-4} \bfseries},beforeskip=0pt,afterskip=0pt}} %section样式设置

\lstset{
    language=Python,       % 设置语言为 Python
    basicstyle=\ttfamily,  % 基本样式
    keywordstyle=\color{blue},  % 关键字颜色
    commentstyle=\color{gray},  % 注释颜色
    stringstyle=\color{red},    % 字符串颜色
    numbers=left,        % 显示行号
    numberstyle=\tiny,   % 行号字体
    frame=single,        % 给代码加上边框
    breaklines=true      % 自动换行
    }
%---------------------------------------------------------------------%
\title{\heiti \zihao{-2}\LaTeX 自制模板}%题目,黑体字,小二号大小
\author{\kaishu \zihao{-4}TYH\\ \songti \zihao{-5} 此处是作者简介:Ningbo University,226002262,major in mathmatic and applied mathmatic}%作者,楷体字,小四大小,作者简介宋体字,小五大小
\date{2024/7/11}%日期
%-----------------------------正文开始---------------------------------%
\begin{document}%开始
\maketitle%加上此指令才会显示文章标题,作者,日期
\vspace{-10pt}%减少作者与标题的间距
\newpage
\begin{center}
	\tableofcontents%生成目录并居中,目录往往要编译两次才会正确显示
\end{center}
\newpage
\begin{center}
	\addcontentsline{toc}{section}{摘要}%将摘要添加到目录中
	\section*{摘要}
\end{center}

此处是摘要,换行用双反斜杠或者空一行,双反斜杠换行没有首行缩进,空一行换行有首行缩进,也就是分段操作。\\
\heiti 关键词:\songti 关键词1,关键词2
\newpage
\section{第一节一级标题}%第一节一级标题
\songti \zihao{-4} 内容1%设置正文部分字体为宋体,小四大小
\subsection{第一节二级标题} \label{第一节二级标题}
内容1.1
\subsubsection{第一节三级标题}
内容1.1.1
\subsection{假设}
(1)假设一:

(2)假设二:
\section{第二节一级标题}
内容2
\subsection{第二节二级标题}
内容2.1
\subsubsection{第二节三级标题}
内容2.1.1
\subsection{第二节二级标题}
内容2.2
\section{图片插入}
\begin{figure}[htbp]
	\centering
	\includegraphics[width=0.65\linewidth]{Cut_Method_Glass}
	\caption{图名}\label{图名}
\end{figure}
\section{表格插入}
\begin{table}[htbp]%h:here当前位置,t:top页面顶部,b:bottom页面底部,p:page另放一页
	\caption{表名}\label{表名}
	\centering
	\begin{tabular}{p{3cm}<{\centering} p{3cm}<{\centering} p{3cm}<{\centering} p{3cm}<{\centering}}
		%p{3cm}规定列宽,\centering规定居中对其
		\toprule[1.5pt]
		1&2&3&4\\
		\midrule[0.75pt]
		内容&内容&内容&内容\\
		内容&内容&内容&内容\\
		内容&内容&内容&内容\\
		\bottomrule[1pt]
	\end{tabular}
\end{table}
\section{数学公式}
\subsection{行内公式}
行内公式,$a+b=c$,希腊字母$\alpha\beta\gamma$,定积分$\int_{a}^{b} f(x)dx$,不定积分$\int f(x)dx$,
\subsection{行间公式}
\begin{equation}
	S_{n}=\sum_{k=1}^{n} x_{k} \label{(1)}
\end{equation}
\begin{equation}
	\alpha\left( x+y\right) =z
\end{equation}
方程组:
\begin{equation}
	\begin{cases}
		x_{1}^{2}+y_{1}^2=a\\
		x_{1}-y_{1}=b\\
		x_{1}+y_{1}=c
	\end{cases}
\end{equation}
分段函数:
\begin{equation}
	f(x)=
	\begin{cases}
		f_{1}(x) &x>a\\
		f_{2}(x) &x<=a
	\end{cases}
\end{equation}
长公式:
\begin{equation}
	\begin{split}
		f(x)&=1000000a+2000000b+30000c+400000d\\
		&=1000000\alpha+200000\beta+3000000\gamma+40000\delta\\
		&=\zeta+\eta+\theta+\lambda
	\end{split}
\end{equation}
矩阵:
\begin{equation}
	\left[ 
	\begin{array}{c}
		\varphi\\
		\theta\\
		\psi
	\end{array}\right] =
	\left[
	\begin{array}{ccc}
		a_{11}&a_{12}&a_{13}\\
		a_{21}&a_{22}&a_{23}\\
		a_{31}&a_{32}&a_{33}
	\end{array}\right] 
	\left[\begin{array}{c}
		x_{1}\\
		x_{2}\\
		x_{3} 
	\end{array}\right] 
\end{equation}
高维矩阵:
\begin{equation}
	\left[ \begin{array}{cccc}
		a_{11}&a_{12}&\cdots&a_{1n}\\
		a_{21}&a_{22}&\cdots&a_{2n}\\
		\vdots&\vdots&{}&\vdots\\
		a_{n1}&a_{n2}&\cdots&a_{nn}
	\end{array}\right] 
\end{equation}
%-------------------------------参考文献-----------------------------%
\begin{thebibliography}{99}
	\addcontentsline{toc}{section}{参考文献}%将参考文献添加到目录中
	\bibitem{参考文献1}这是参考文献1
	
	\bibitem{参考文献2}这是参考文献2
	
	\bibitem{参考文献3}这是参考文献3
\end{thebibliography}
%--------------------------------附录--------------------------------%
\section*{附录A}
\addcontentsline{toc}{section}{附录A}%将附录A添加到目录中
外部网址链接:\url{www.bing.com}

文内交叉引用测试:\ref{第一节二级标题}

公式引用测试\ref{(1)}

脚注测试\footnote{这是脚注}

表格引用测试 表\ref{表名}

图片引用测试 图\ref{图名}

参考文献测试\textsuperscript{\cite{参考文献1}}

参考文献测试\textsuperscript{\cite{参考文献1}\cite{参考文献2}}
\end{document}%结束
